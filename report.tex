\documentclass{article}
    \usepackage[utf8]{inputenc}

    \title{Serial Code Optimisation}
    \author{David Sharp:ds16797 Candidate 36688}
    \date{October 2018}
    
    \begin{document}

    \maketitle
    \section{Introduction}
    In summary, I managed to partially optimise the 5-point stencil but not to the benchmarks listed in the assignment.
    In this report I will detail the optimisation techniques that worked for me and also outline the optimisations I hoped to utilise
    but couldn't fully implement.
    \section{Optimisations}
    \subsection{Initial State}
    \begin{center}
    \begin{tabular}{ c c c c }
    Compiler settings & 1024 image & 4096 image & 8000 image \\
    No flags gcc4.8.4 & 8.23s & 318s & 735s \\
    No flags gcc7.1.0 & 8.20s & 433s & 577s \\
    -O3 gcc7.1.0 & 6.81 & 121s & 362s \\
    -Ofast gcc7.1.0 & 2.34s & 108s & 125s \\
    No flags icc16 & 3.40s & 49.3s & 176s \\
    -xHost icc16 & 3.42s & 49.3s & 175s \\
    -Ofast icc16 & 2.13s & 93.6s & 105s
    \end{tabular}
    \end{center}
    These initial optimisations already show some interesting results, with gcc7.1.0 not being consistently faster across different scales and the Intel
    compiler icc version 16 being significantly faster than gcc7.1.0.
    -O3 and -Ofast each gave significant speedups on gcc, while -xHost - designed to adapt the code for the specific Intel processor - offered no speedup
    over the icc defaults.
    
    \subsection{Profiling}
    After getting the initial simple speedup from compiler flags, I ran gprof to get an intuition for where the majority of the runtime is located.
    The results were as following on the 1024 image.
    \begin{center}
    \begin{tabular}{ c c c c }
    Percentage Time & Self Seconds & Calls & Name \\
    99.84 & 8.19 & 200 & stencil \\
    0.37 & 0.03 & 1 & init\_image \\
    0.12 & 0.01 & 1 & output\_image \\
    0.00 & 0.00 & 2 & wtime
    \end{tabular}
    \end{center}
    From the gprof profiling, it's very clear that the stencil function is the limiter on the code speed, since it is run 200 times. As such
    it makes the most sense to focus optimisation on this one function.
    \subsection{Applied Optimisation}
    My first thought for optimising the code was to reduce the math operation for each part of the stencil, changing the value change by multiplication 
    then division by simplifying into a single multiplication. While this change may be already carried out by the more rigorous compiler flags, it makes
    the code more readable and from my own research multiplication is at least as fast as division in many languages.
    Following this 
    
    Following this change I 
    \end{document}